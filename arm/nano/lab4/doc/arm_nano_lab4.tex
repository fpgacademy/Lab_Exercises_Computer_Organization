\documentclass[epsfig,10pt,fullpage]{article}

\newcommand{\LabNum}{4}
\newcommand{\CommonDocsPath}{../../../../common/docs}
\input{\CommonDocsPath/preamble.tex}

\begin{document}

\centerline{\huge Computer Organization}
~\\
\centerline{\huge Laboratory Exercise \LabNum}
~\\
\centerline{\large Input/Output in an Embedded System}
~\\

\noindent
The purpose of this exercise is to investigate the use of devices that provide input and
output capabilities for a processor. There are two basic techniques for dealing with I/O devices:
program-controlled polling and interrupt-driven approaches.  You will use the polling approach 
in this exercise, writing programs in the ARM* assembly language.  We assume that your programs 
will be run on an ARM processor in the DE10-Nano Computer, implemented in the DE10-Nano
board (or a similar DE-Series board).  Parallel port interfaces, as well as a timer module, 
will be used as examples of I/O hardware.

~\\
\noindent
A parallel port provides for data transfer in either the input or output direction.  The 
transfer of data is done in parallel and it may involve from 1 to 32 bits. The number of
bits, $n$, and the type of transfer depend on the specifications of the specific parallel port 
being used.  The parallel port interface can contain the four registers shown in
Figure~\ref{fig:parallel}.

\begin{figure}[htb]
	\begin{center}
	\includegraphics[scale=1]{figures/figureparallel.pdf}
	\end{center}
	\caption{Registers in the parallel port interface.}
\label{fig:parallel}
\end{figure}

\noindent
Each register is $n$ bits long. The registers have the following purpose:
\begin{itemize}
\item {\it Data} register: holds the $n$ bits of data that are transferred between the parallel 
port and the ARM processor. It can be implemented as an input, output, or a bidirectional register.
\item {\it Direction} register: defines the direction of transfer for each of the $n$
data bits when a bidirectional interface is generated.
\item {\it Interrupt-mask} register: used to enable interrupts from the
input lines connected to the parallel port.
\item {\it Edge-capture} register: indicates when a change of logic value is detected in 
the signals on the input lines connected to the parallel port. Once a bit in the edge
capture register becomes asserted, it will remain asserted. An edge-capture bit can be
de-asserted by writing to it using the ARM processor.
\end{itemize}
\noindent
Not all of these registers are present in some parallel ports. For example,
the {\it Direction} register is included only when a bidirectional interface is specified.
The {\it Interrupt-mask} and {\it Edge-capture} registers must be included if
interrupt-driven input/output is used.

~\\
The parallel port registers are memory mapped, starting at a specific {\it base} address.
This base address is the address of the {\it Data} 
register in the parallel port. The addresses of the other three registers have offsets 
of 4, 8, or 12 bytes (1, 2, or 3 words) from this base address. The DE10-Nano Computer has 
parallel ports connected to SW slide switches, KEY pushbuttons, and LEDs.

~\\
\noindent
{\bf Part I}
~\\
~\\
\noindent
Write an ARM assembly language program that displays a binary number on the green 
lights {\it LED}$_{3-0}$ on the DE10-Nano board. The other lights {\it LED}$_{7-4}$
should be off. 

~\\
\noindent
The parallel port in the DE10-Nano Computer connected to the green lights
{\it LED}$_{7-0}$ is memory mapped at the address {\sf 0xFF200000}. 
Figure~\ref{fig:LED} shows how the LEDs are connected to the parallel port bits.  

\begin{figure}[htb]
	\begin{center}
	\includegraphics[scale=1]{figures/figureLED.pdf}
	\end{center}
	\caption{The parallel port connected to the green lights {\it LED}$_{7-0}$.}
\label{fig:LED}
\end{figure}

~\\
\noindent
Perform the following:

\begin{enumerate}
\item If {\it KEY}$_0$ is pressed on the board, you should set the number displayed on the
LEDs to 0. If {\it KEY}$_1$ is pressed and {\it SW}$_0$ is high, then increment the displayed
number to a maximum of 9. If {\it KEY}$_1$ is pressed and {\it SW}$_0$ is low, then decrement the
number to a minimum of 0. You can think of the displayed binary number as a {\it decimal digit},
since its value is restricted to be from 0 to 9 (\green{0000} to \green{1001}). The 
parallel port connected to the KEY pushbuttons has the base address
{\sf 0xFF200050}, as illustrated in Figure~\ref{fig:KEY}. The parallel port connected to
the SW slide switches has the base address {\sf 0xFF200040}, as illustrated in Figure~\ref{fig:SW}.
In your program, use polled I/O to read the {\it Data} registers of the {\it KEY} and {\it SW}
ports to check the status of the buttons and switches. When you are not pressing 
any {\it KEY} the {\it Data} register provides~0, and when you press {\it KEY}$_i$ the 
{\it Data} register provides the value 1 in bit position $i$. Once a button-press is detected,
be sure that your program waits until the button is released. You should not use the 
{\it Interruptmask} or {\it Edgecapture} registers for this part of the exercise.

\begin{figure}[htb]
	\begin{center}
	\includegraphics[scale=1]{figures/figureKEY.pdf}
	\end{center}
	\caption{The parallel port connected to the KEY pushbuttons.}
\label{fig:KEY}
\end{figure}

\begin{figure}[htb]
	\begin{center}
	\includegraphics[scale=1]{figures/figureSW.pdf}
	\end{center}
	\caption{The parallel port connected to the SW slide switches.}
\label{fig:SW}
\end{figure}

\item Create a new folder to hold your solution for this part. Create a
file called {\it part1.s} and type your assembly language code into this file.

\item
Make a new Monitor Program project in the folder where you stored the {\it part1.s}
file. Use the DE10-Nano Computer for this project, and select ARM as the target
processor architecture.

\item
Compile, download, and test your program. 
\end{enumerate}

~\\
\noindent
{\bf Part II}
~\\
~\\
\noindent
In Part 1 you displayed a 4-bit binary number on the green LEDs. Since the value of the number was
restricted to be from 0 to 9, it could be considered as a {\it decimal digit}. For this part, 
you are to display {\it two} 4-bit binary values on the LEDs, representing {\it two} decimal digits:
the most-significant decimal digit should be shown on {\it LED}$_{7-4}$, and the least-significant
digit on {\it LED}$_{3-0}$. The decimal digits should function as a {\it counter} that takes
values from 00 to 99 (\green{00000000} to \green{10011001}). 
Write an ARM assembly language program that implements and displays
this counter. The counter should be incremented approximately every $0.25$ seconds. When the 
counter reaches the value 99, it should start again at 00. The counter should stop/start 
when {\it KEY}$_0$ or {\it KEY}$_1$ is pressed.

~\\
\noindent
To achieve a delay of approximately 0.25 seconds, use a delay-loop in your assembly language
code. A suitable example of such a loop is shown below.

\begin{minipage}[t]{12.5 cm}
\begin{lstlisting}[style=defaultArmStyle]
DO_DELAY: 	LDR		R7, =200000000 		// delay counter
SUB_LOOP: 	SUBS	R7, R7, #1
			BNE 	SUB_LOOP
\end{lstlisting}
\end{minipage}

~\\
\noindent
To avoid ``missing'' any button presses while the processor is executing the delay loop, you
should use the {\it Edgecapture} register in the {\it KEY} port, shown in Figure~\ref{fig:KEY}.
When a pushbutton is pressed, the corresponding bit in the {\it Edgecapture} register is
set to 1; it remains set until your program resets it to 0 by writing into the register.
~\\
~\\
\noindent
Perform the following:

\begin{enumerate}
\item Create a new folder to hold your solution for this part. Create a
file called {\it part2.s} and type your assembly language code into this file.

\item
Make a new Monitor Program project in the folder where you stored the {\it part2.s}
file. Use the DE10-Nano Computer for this project, and select ARM as the target
processor architecture.

\item
Compile, download, and test your program. 
\end{enumerate}

~\\
\noindent
{\bf Part III}
~\\
~\\
\noindent
In Part II you used a delay loop to cause the ARM processor to wait for approximately 0.25 
seconds. The processor loaded a large value into a register before the loop, and then 
decremented that value until it reached 0.  In this part you are to modify your code so that a
hardware timer is used to measure an exact delay of 0.25 seconds. You should use polled I/O to
cause the ARM processor to wait for the timer.

~\\
\noindent
The DE10-Nano Computer includes a number of hardware timers. For this exercise use the timer
called the ARM A9 {\it Private Timer}. As shown in Figure~\ref{fig:timer} this timer has 
four registers, starting at the base address {\sf 0xFFFEC600}. To use the timer you need
to write a suitable value into the {\it Load} register. Then, you need to set the enable
bit $E$ in the {\it Control} register to 1, to start the timer. The timer starts counting from
the initial value in the {\it Load} register and counts down to 0 at a frequency of 200 MHz. 
The counter will automatically reload the value in the {\it Load} register and continue counting 
if the $A$ bit in the {\it Control} register is set to 1.  When it reaches 0, the timer sets 
the $F$ bit in the {\it Interrupt status} register to 1.
You should poll this bit in your program to cause the ARM processor to wait for the timer. 
To reset the $F$ bit to 0 you have to write the value \texttt{1} into this bit-position. 

~\\
\begin{figure}[htb]
	\begin{center}
	\includegraphics[scale=1]{figures/figuretimer.pdf}
	\end{center}
	\caption{The ARM A9 Private Timer registers.}
\label{fig:timer}
\end{figure}

~\\
\noindent
Perform the following:

\begin{enumerate}
\item Make a new folder to hold your solution for this part. Create a file called {\it part3.s} 
and type your assembly language code into this file.
\item Make a new Monitor Program project for this part of the exercise, and then compile, download, 
and test your program. 
\end{enumerate}

~\\
~\\
\noindent
{\bf Part IV}
~\\
~\\
\noindent
In this part you are to write an assembly language program that implements a real-time clock. 
Display the time on the Monitor Program's Terminal window in the format 
\blue{MM}:\blue{SS}, where \blue{\it MM} are minutes and \blue{\it SS} are seconds.
Measure time intervals of 1 second in your program by using polled I/O with the ARM A9 
Private Timer.  You should be able to stop/run the clock by pressing any KEY pushbutton.
When the clock reaches \blue{59}:\blue{99}, it should wrap around to \blue{00}:\blue{00}.

~\\
\noindent
Perform the following:

\begin{enumerate}
\item
Make a new folder for your solution to this part. Create a file called {\it part4.c} for 
your assembly language code. 
\item
Make a new Monitor Program project in the folder created for {\it part4.c}. In the screen 
shown in Figure~\ref{fig:terminal}, make sure to select {\sf JTAG\_UART\_for\_ARM\_0} as 
the {\it Terminal device}. Otherwise, no character output will appear on the Terminal window.
Refer to Exercise 2, Part IV, for information on using the JTAG* UART to communicate with 
the Monitor Program's Terminal window.  

\begin{figure}[htb]
	\begin{center}
	\includegraphics[scale=0.58]{figures/terminal.png}
	\end{center}
	\vspace{-0.25cm}\caption{Specifying the {\it Terminal device}.}
\label{fig:terminal}
\end{figure}

\item
You may wish to make use of the following text 
strings. The first one clears the Terminal window, and the second one returns the "cursor" to the 
upper-left corner of the window:
~\\
\begin{minipage}[t]{12.5 cm}
\begin{lstlisting}[style=defaultArmStyle]
CLR_SCRN: 	.asciz "\033[2J"
HOME: 		.asciz "\033[H"
\end{lstlisting}
\end{minipage}
\item
Test your program to see that it displays the real-time clock on the Terminal window, and that the
clock can be stopped/run by pressing the KEY pushbuttons.
\end{enumerate}


\input{\CommonDocsPath/copyright.tex}
\end{document}
