\documentclass[epsfig,10pt,fullpage]{article}

\newcommand{\LabNum}{4}
\newcommand{\CommonDocsPath}{../../../../common/docs}
\input{\CommonDocsPath/preamble.tex}

\begin{document}

\centerline{\huge Computer Organization}
~\\
\centerline{\huge Laboratory Exercise \LabNum}
~\\
\centerline{\large Input/Output in an Embedded System}
~\\

\noindent
The purpose of this exercise is to investigate the use of devices that provide input and
output capabilities for a processor. There are two basic techniques for dealing with I/O devices:
program-controlled polling and interrupt-driven approaches.  You will use the polling approach 
in this exercise, writing programs in the ARM* assembly language. 
Your programs will be executed on an ARM processor in either the DE1-SoC or DE10-Standard
computer system.  Parallel port interfaces, 
as well as a timer module, will be used as examples of I/O hardware.

~\\
\noindent
A parallel port provides for data transfer in either the input or output direction.  The 
transfer of data is done in parallel and it may involve from 1 to 32 bits. The number of
bits, $n$, and the type of transfer depend on the specifications of the specific parallel port 
being used.  The parallel port interface can contain the four registers shown in
Figure~\ref{fig:parallel}.

\begin{figure}[htb]
	\begin{center}
	\includegraphics[scale=1]{figures/figureparallel.pdf}
	\end{center}
	\caption{Registers in the parallel port interface.}
\label{fig:parallel}
\end{figure}

\noindent
Each register is $n$ bits long. The registers have the following purpose:
\begin{itemize}
\item {\it Data} register: holds the $n$ bits of data that are transferred between the parallel 
port and the ARM processor. It can be implemented as an input, output, or a bidirectional register.
\item {\it Direction} register: defines the direction of transfer for each of the $n$
data bits when a bidirectional interface is generated.
\item {\it Interrupt-mask} register: used to enable interrupts from the
input lines connected to the parallel port.
\item {\it Edge-capture} register: indicates when a change of logic value is detected in 
the signals on the input lines connected to the parallel port. Once a bit in the edge
capture register becomes asserted, it will remain asserted. An edge-capture bit can be
de-asserted by writing to it using the ARM processor.
\end{itemize}
\noindent
Not all of these registers are present in some parallel ports. For example,
the {\it Direction} register is included only when a bidirectional interface is specified.
The {\it Interrupt-mask} and {\it Edge-capture} registers must be included if
interrupt-driven input/output is used.

~\\
The parallel port registers are memory mapped, starting at a specific {\it base} address.
The base address becomes the address of the {\it Data} 
register in the parallel port. The addresses of the other three registers have offsets 
of 4, 8, or 12 bytes (1, 2, or 3 words) from this base address. In DE-series computers parallel 
ports are used to connect to SW slide switches, KEY pushbuttons, LEDs, and seven-segment displays.

~\\
\noindent
{\bf Part I}
~\\
~\\
\noindent
Write an ARM assembly language program that displays a decimal digit on the seven-segment
display {\it HEX}0. The other seven-segment 
displays {\it HEX}$5-1$ should be blank.  

~\\
\noindent
The DE1-SoC, and DE10-Standard, Computer contains a parallel port connected to the seven-segment 
displays {\it HEX}$3-0$. The port is memory mapped at the base address {\sf 0xFF200020}. A
second parallel port is connected to {\it HEX}$5-4$, at the base address {\sf 0xFF200030}. 
Figure~\ref{fig:HEX} shows how the display segments are connected to the parallel ports.  

~\\
\begin{figure}[htb]
	\begin{center}
	\includegraphics[scale=1]{figures/figureHEX.pdf}
	\end{center}
	\caption{The parallel ports connected to the seven-segment displays {\it HEX}$5-0$.}
\label{fig:HEX}
\end{figure}

\newpage
Perform the following:

\begin{enumerate}
\item If {\it KEY}$_0$ is pressed on the board, you should set the number displayed on 
{\it HEX}0 to 0. If {\it KEY}$_1$ is pressed then increment the displayed number, and 
if {\it KEY}$_2$ is pressed then decrement the number. Pressing {\it KEY}$_3$ should 
blank the display, and pressing any other KEY after that should return the display to 0.
The parallel port connected to the pushbutton {\it KEYs} has the base address
{\sf 0xFF200050}, as illustrated in Figure~\ref{fig:KEY}. In your program, use polled I/O 
to read the {\it Data} register to see when a button is being pressed. When you are not pressing 
any {\it KEY} the {\it Data} register provides~0, and when you press {\it KEY}$_i$ the 
{\it Data} register provides the value 1 in bit position $i$. Once a button-press is detected,
be sure that your program waits until the button is released. You should not use the 
{\it Interruptmask} or {\it Edgecapture} registers for this part of the exercise.

~\\
\begin{figure}[htb]
	\begin{center}
	\includegraphics[scale=1]{figures/figureKEY.pdf}
	\end{center}
	\caption{The parallel port connected to the pushbutton {\it KEYs}.}
\label{fig:KEY}
\end{figure}

\item Create a new folder to hold your solution for this part. Create a
file called {\it part1.s} and type your assembly language code into this file.
You may want to refer to a discussion, and examples of assembly-language code,
in Part IV of Lab Exercise 2 about displaying numbers on seven-segment displays.

\item
Make a new Monitor Program project in the folder where you stored the {\it part1.s}
file. Use either the DE1-SoC or DE10-Standard pre-built computer system for your project, 
and select ARM as the target processor architecture.

\item
Compile, download, and test your program. 
\end{enumerate}

~\\
\noindent
{\bf Part II}
~\\
~\\
\noindent
Write an ARM assembly language program that displays a two-digit decimal counter on the 
seven-segment displays {\it HEX}$1-0$. The counter should be incremented approximately
every $0.25$ seconds. When the counter reaches the value 99, it should start again at 0.
The counter should stop/start when any pushbutton {\it KEY} is pressed.

~\\
\noindent
To achieve a delay of approximately 0.25 seconds, use a delay-loop in your assembly language
code. A suitable example of such a loop is shown below.

\noindent
~\\
\begin{lstlisting}[style=defaultArmStyle]
DO_DELAY: 	LDR 	R7, =200000000 		// delay counter
SUB_LOOP: 	SUBS 	R7, R7, #1
			BNE 	SUB_LOOP
\end{lstlisting}

~\\
\noindent
To avoid ``missing'' any button presses while the processor is executing the delay loop, you
should use the {\it Edgecapture} register in the {\it KEY} port, shown in Figure~\ref{fig:KEY}.
When a pushbutton is pressed, the corresponding bit in the {\it Edgecapture} register is
set to 1; it remains set until your program resets it back to 0 by writing into the register.

~\\
\noindent
Perform the following:

\begin{enumerate}
\item Create a new folder to hold your solution for this part. Create a
file called {\it part2.s} and type your assembly language code into this file.

\item
Make a new Monitor Program project in the folder where you stored the {\it part2.s}
file. Use the appropriate pre-built computer system for your DE-series board for this project, 
and select ARM as the target processor architecture.

\item
Compile, download, and test your program. 
\end{enumerate}

~\\
\noindent
{\bf Part III}
~\\
~\\
\noindent
In Part II you used a delay loop to cause the ARM processor to wait for approximately 0.25 
seconds. The processor loaded a large value into a register before the loop, and then 
decremented that value until it reached 0.  In this part you are to modify your code so that a
hardware {\it timer} is used to measure an exact delay of 0.25 seconds. 
You should use polled I/O to
cause the ARM processor to wait for the timer.

~\\
\noindent
The DE-series computer systems include a number of hardware timers. For this exercise use the timer
called the ARM A9 {\it Private Timer}. As shown in Figure~\ref{fig:timer} this timer has 
four registers, starting at the base address {\sf 0xFFFEC600}. To use the timer you need
to write a suitable value into the {\it Load} register. Then, you need to set the enable
bit $E$ in the {\it Control} register to 1, to start the timer. The timer starts counting from
the initial value in the {\it Load} register and counts down to 0 at a frequency of 200 MHz. 
The counter will automatically reload the value in the {\it Load} register and continue counting 
if the $A$ bit in the {\it Control} register is set to 1.  When it reaches 0, the timer sets 
the $F$ bit in the {\it Interrupt status} register to 1.
You should poll this bit in your program to cause the ARM processor to wait for the timer. 
To reset the $F$ bit to 0 you have to write the value \texttt{1} into this bit-position. 

\begin{figure}[htb]
	\begin{center}
	\includegraphics[scale=1]{figures/figuretimer.pdf}
	\end{center}
	\caption{The ARM A9 Private Timer registers.}
\label{fig:timer}
\end{figure}

~\\
\noindent
Perform the following:

\begin{enumerate}
\item
Make a new folder to hold your solution for this part. Create a file called {\it part3.s} and 
type your assembly language code into this file.  
\item Make a new Monitor Program project for this 
part of the exercise, and then compile, download, and test your program. 
\end{enumerate}

\newpage
\noindent
{\bf Part IV}
~\\
~\\
\noindent
In this part you are to write an assembly language program that implements a real-time clock. 
Display the time on the seven-segment displays {\it HEX}$3-0$ in the format 
\red{SS}:\red{DD}, where \red{\it SS} are seconds and \red{\it DD} are hundredths of a second.
Measure time intervals of 0.01 seconds in your program by using polled I/O with the ARM A9 
Private Timer.  You should be able to stop/run the clock by pressing any pushbutton {\it KEY}.
When the clock reaches \red{59}:\red{99}, it should wrap around to \red{00}:\red{00}.

~\\
\noindent
Perform the following:

\begin{enumerate}
\item Make a new folder to hold your solution for this part. Create a
file called {\it part4.s} and type your code into this file.  
\item Make a new Monitor Program 
project for this part of the exercise, and then compile, download, and test your program. 
\end{enumerate}


\input{\CommonDocsPath/copyright.tex}
\end{document}
