\documentclass[epsfig,10pt,fullpage]{article}

\newcommand{\LabNum}{4}
\newcommand{\CommonDocsPath}{../../../../common/docs}
\input{\CommonDocsPath/preamble.tex}

\begin{document}

\centerline{\huge Computer Organization}
~\\
\centerline{\huge Laboratory Exercise \LabNum}
~\\
\centerline{\large Input/Output in an Embedded System}
~\\

\noindent
The purpose of this exercise is to investigate the use of devices that provide input and
output capabilities for a processor. There are two basic techniques for dealing with I/O devices:
program-controlled polling and interrupt-driven approaches.  You will use the polling approach 
in this exercise, writing programs in the Nios\textsuperscript{\textregistered} II assembly language. 
Your programs will be executed on an Nios II processor in the DE0-Nano,
DE0-Nano-SoC, or DE10-Nano Computer, implemented on an Intel\textsuperscript{\textregistered} DE-Series board.  Parallel port interfaces, 
as well as a timer module, will be used as examples of I/O hardware.

~\\
\noindent
A parallel port provides for data transfer in either the input or output direction.  The 
transfer of data is done in parallel and it may involve from 1 to 32 bits. The number of
bits, $n$, and the type of transfer depend on the specifications of the specific parallel port 
being used.  The parallel port interface can contain the four registers shown in
Figure~\ref{fig:parallel}.

\begin{figure}[htb]
	\begin{center}
	\includegraphics[scale=1]{figures/figureparallel.pdf}
	\end{center}
	\caption{Registers in the parallel port interface.}
\label{fig:parallel}
\end{figure}

\noindent
Each register is $n$ bits long. The registers have the following purpose:
\begin{itemize}
\item {\it Data} register: holds the $n$ bits of data that are transferred between the parallel 
port and the processor. It can be implemented as an input, output, or a bidirectional register.
\item {\it Direction} register: defines the direction of transfer for each of the $n$
data bits when a bidirectional interface is generated.
\item {\it Interrupt-mask} register: used to enable interrupts from the
input lines connected to the parallel port.
\item {\it Edge-capture} register: indicates when a change of logic value is detected in 
the signals on the input lines connected to the parallel port. Once a bit in the edge
capture register becomes asserted, it will remain asserted. An edge-capture bit can be
de-asserted by writing to it using the Nios II processor.
\end{itemize}
\noindent
Not all of these registers are present in some parallel ports. For example,
the {\it Direction} register is included only when a bidirectional interface is specified.
The {\it Interrupt-mask} and {\it Edge-capture} registers must be included if
interrupt-driven input/output is used.

~\\
The parallel port registers are memory mapped, starting at a specific {\it base} address.
The base address has to be a multiple of four if the parallel port is to be accessed using
word accesses from the Nios II processor. The base address becomes the address of the {\it Data} 
register in the parallel port. The addresses of the other three registers have offsets 
of 4, 8, or 12 bytes (1, 2, or 3 words) from this base address.
The DE-Series Computer Systems have parallel ports connected to slider switches, pushbutton KEYs,
and LEDs.

~\\
\noindent
{\bf Part I}
~\\
~\\
\noindent
Write an Nios II assembly language program that displays a decimal digit on the green 
lights {\it LED}$_{3-0}$ on the DE-Series board. The other lights {\it LED}$_{7-4}$
should be off. 

~\\
\noindent
The parallel port in the pre-built computer systems connected to the green lights
{\it LED}$_{7-0}$ is memory mapped at the address {\sf 0xFF200000}. 
Figure~\ref{fig:LED2} shows how the LEDs are connected to the parallel ports.  

\begin{figure}[htb]
	\begin{center}
	\includegraphics[scale=1]{figures/DE0_Nano_SoC/figureLED.pdf}
	\end{center}
	\caption{The parallel port connected to the green lights {\it LED}$_{7-0}$.}
\label{fig:LED2}
\end{figure}

\noindent
If {\it KEY}$_0$ is pressed on the board, you should set the number displayed on the
LEDs to 0. If {\it KEY}$_1$ is pressed and {\it SW}$_0$ is high, then increment the displayed
number to a maximum of 9. If {\it KEY}$_1$ is pressed and {\it SW}$_0$ is low, then decrement the
number to a minimum of 0. The parallel port connected to the pushbutton {\it KEYs} has the base address
{\sf 0xFF200050} in the computer system, as illustrated 
in Figure~\ref{fig:KEY2}.
The parallel port connected to the slider switches {\it SW} has the base address {\sf 0xFF200040} 
in the pre-built computer systems, as illustrated in Figure~\ref{fig:SW2}.
In your program, use polled I/O to read the {\it Data} registers of the {\it KEY} and {\it SW}
ports to check the status of the buttons and switches. When you are not pressing 
any {\it KEY} the {\it Data} register provides~0, and when you press {\it KEY}$_i$ the 
{\it Data} register provides the value 1 in bit position $i$. Once a button-press is detected,
be sure that your program waits until the button is released. You should not use the 
{\it Interruptmask} or {\it Edgecapture} registers for this part of the exercise.

\begin{figure}[H]
	\begin{center}
	\includegraphics[scale=1]{figures/DE0_Nano_SoC/figureKEY.pdf}
	\end{center}
	\caption{The parallel port connected to the pushbutton {\it KEYs}.}
\label{fig:KEY2}
\end{figure}

\begin{figure}[H]
	\begin{center}
	\includegraphics[scale=1]{figures/DE0_Nano_SoC/figureSW.pdf}
	\end{center}
	\caption{The parallel port connected to the slider switches {\it SW}.}
\label{fig:SW2}
\end{figure}

~\\
\noindent
Perform the following:

\begin{enumerate}
\item Create a new folder to hold your Monitor Program project for this part. Create a
file called {\it part1.s} and type your assembly language code into this file.

\item
Make a new Monitor Program project in the folder where you stored the {\it part1.s}
file. Use the DE0-Nano, DE0-Nano-SoC, or DE10-Nano Computer for this project, and select 
the Nios II as the target processor architecture.

\item
Compile, download, and test your program. 
\end{enumerate}

~\\
\noindent
{\bf Part II}
~\\
~\\
\noindent
Write a Nios II assembly language program that displays a two-digital decimal counter on the 
green LEDs. Show the most-significant decimal digit on {\it LED}$_{7-4}$, and
the least-significant digit on {\it LED}$_{3-0}$. The counter should be incremented approximately
every $0.25$ seconds. When the counter reaches the value 99, it should start again at 0.
The counter should stop/start when any pushbutton {\it KEY} is pressed.

~\\
\noindent
To achieve a delay of approximately 0.25 seconds, use a delay-loop in your assembly language
code. A suitable example of such a loop is shown below.

\begin{minipage}[t]{12.5 cm}
\begin{lstlisting}[style=defaultNiosStyle]
DO_DELAY: 	movia 	r7, 10000000 	#delay counter
SUB_LOOP: 	subi 	r7, r7, 1
			bne 	r7, r0, SUB_LOOP
\end{lstlisting}
\end{minipage}

~\\
\noindent
To avoid "missing" any button presses while the processor is executing the delay loop, you
should use the {\it Edgecapture} register in the {\it KEY} port, shown in Figure~\ref{fig:KEY2}.
When a pushbutton is pressed, the corresponding bit in the {\it Edgecapture} register is
set to 1, and it remains set, until a 1 is written into to the register, which clears the 
{\it Edgecapture} register.

\noindent
Perform the following:

\begin{enumerate}
\item Create a new folder to hold your Monitor Program project for this part. Create a
file called {\it part2.s} and type your assembly language code into this file.

\item
Make a new Monitor Program project in the folder where you stored the {\it part2.s}
file. Use the DE0-Nano, DE0-Nano-SoC, or DE10-Nano Computer for this project, and select 
the Nios II as the target processor architecture.

\item
Compile, download, and test your program. 
\end{enumerate}

~\\
\noindent
{\bf Part III}
~\\
~\\
\noindent
In Part II you used a delay loop to cause the Nios II processor to wait for approximately 0.25 
seconds. The processor loaded a large value into a register before the loop, and then 
decremented that value until it reached 0.  In this part you are to modify your code so that a
hardware timer is used to measure an exact delay of 0.25 seconds. You should use polled I/O to
cause the Nios II processor to wait for the timer.

~\\
\noindent
The pre-built computer systems includes a number of hardware timers. For this exercise use the timer
called the {\it Interval Timer}. As shown in Figure~\ref{fig:timer2}, this timer has 
six, 16-bit registers, starting at the base address {\sf 0xFF202000}. 
To use the timer you should
write a suitable value into the {\it Counter start value (low)} and {\it Counter start value (high)} 
registers, where the {\it low} register holds the least significant 16 bits of the timeout
period, and the {\it high} register holds the most significant 16 bits of the timeout period. Then you should clear the 
timeout (TO) bit by writing 0 to the {\it Status register}. The timer should count down
to 0 and then reload the original period, and begin counting again. To achieve this, you must write a 1 to the START and CONT
bits in the {\it Control register}, in Figure~\ref{fig:timer2}. When the timer has finished, the TO bit in
the {\it Status register} will be set to 1. You should poll this bit in your program to cause the Nios II
processor to wait for the timer. To reset the timeout bit, you should write a 0 into the {\it Status register}.

\begin{figure}[htb]
	\begin{center}
	\includegraphics[scale=0.8]{figures/DE0_Nano_SoC/fig_interval_port.pdf}
	\end{center}
	\caption{The computer system's interval timer registers.}
\label{fig:timer2}
\end{figure}

~\\
\noindent
Make a new folder to hold your Monitor Program project for this part. Create a
file called {\it part3.s} and type your assembly language code into this file.
Make a new Monitor Program project for this part of the exercise, and then compile, download, 
and test your program. 

~\\
~\\
\noindent
{\bf Part IV}
~\\
~\\
\noindent
In this part you are to write an assembly language program that implements a real-time clock. 
Display the time on the Monitor Program's Terminal window in the format 
\blue{MM}:\blue{SS}, where \blue{\it MM} are minutes and \blue{\it SS} are seconds.
Measure time intervals of 1 second in your program by using polled I/O with the Interval Timer.
You should be able to stop/run the clock by pressing any pushbutton {\it KEY}. 
When the clock reaches \blue{59}:\blue{99}, it should wrap around to \blue{00}:\blue{00}.

~\\
\noindent
Make a new Monitor Program project for this part of the exercise. In the screen shown in Figure
\ref{fig:terminal}, make sure to select {\sf JTAG\_UART} as the {\it Terminal
device}. Otherwise, no character output will appear on the Terminal window.
Refer to Exercise 2, Part IV, for information on using the JTAG* UART to communicate with 
the Monitor Program's Terminal window.  

\begin{figure}[htb]
	\begin{center}
	\includegraphics[scale=0.58]{figures/terminal.png}
	\end{center}
	\vspace{-0.25cm}\caption{Specifying the {\it Terminal device}.}
\label{fig:terminal}
\end{figure}

~\\
\noindent
You may wish to make use of the following text 
strings. The first one clears the Terminal window, and the second one returns the "cursor" to the 
upper-left corner of the window:
\vspace{0.25cm}
~\\
\begin{minipage}[t]{12.5 cm}
\begin{lstlisting}[style=defaultNiosStyle]
CLEAR_SCRN: 	.string "\033[2J"
HOME: 			.string "\033[H"
\end{lstlisting}
\end{minipage}


\input{\CommonDocsPath/copyright.tex}
\end{document}
\end{document}
