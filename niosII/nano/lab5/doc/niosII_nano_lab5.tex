\documentclass[epsfig,10pt,fullpage]{article}

\newcommand{\LabNum}{5}
\newcommand{\CommonDocsPath}{../../../../common/docs}
\input{\CommonDocsPath/preamble.tex}

\begin{document}

\centerline{\huge Computer Organization}
~\\
\centerline{\huge Laboratory Exercise \LabNum}
~\\
\centerline{\large Using Interrupts with Assembly Code}
~\\

\noindent
The purpose of this exercise is to investigate the use of interrupts for the Nios\textsuperscript{\textregistered} II processor, using assembly language
code. To do this exercise you need to be familiar with the exceptions processing mechanisms
for the Nios II processor, which are discussed in the tutorial {\it Nios II Introduction}, available at the Intel\textsuperscript{\textregistered} FPGA
University Program website. You should also read the information on exceptions and interrupts of the DE0-Nano-SoC, DE0-Nano, or DE10-Nano
Computer, depending on the board you are using.
\noindent
~\\
~\\

\section*{Part I}
\addcontentsline{toc}{1}{Part I}
Consider the main program shown in Figure~\ref{fig:code}. The main program needs to set up the stack pointer, configure
the pushbutton KEYs port to generate interrupts, and then enable interrupts in the Nios II processor. You
are to fill in the code that is not shown in the listing.  

~\\
\noindent
The function of your program is to turn on/off the green lights {\it LED}$_1$ and {\it LED}$_0$
when a corresponding pushbutton {\it KEY}$_1$ or {\it KEY}$_0$ is pressed. 
Since the main program simply ''idles'' in an endless loop, as
shown in Figure~\ref{fig:code}, you have to control the LEDs by using an 
interrupt service routine for the pushbutton KEYs.

~\\
\noindent
Perform the following:

\begin{enumerate}
\item Create a new folder to hold your files for this part. Create a
file, such as {\it part1.s}, and copy the assembly language code for the main program,
given in F~\ref{fig:code}, into this file. Create a file {\it exception\_handler.s},
and copy the code given in Figure~\ref{fig:handlers} into it.
Create any other source-code files that you need.

\item 
Figure~\ref{fig:handlers} gives the code required for the Nios~II reset and 
exceptions handlers. The exception handler calls a subroutine {\it KEY\_ISR} to handle interrupts from the {\it KEY} pushbuttons.
Create a file {\it key\_isr.s} and write the code for the {\it KEY\_ISR} interrupt service routine.
Your code should turn on {\it LED}$_1$ display when {\it KEY}$_1$ is
pressed, and then if {\it KEY}$_1$ is pressed again, {\it LED}$_1$ should be turned off . You should
toggle {\it LED}$_1$ on and off each time {\it KEY}$_1$ is pressed. You should toggle {\it LED}$_0$ when {\it KEY}$_0$ is pressed, the same as when {\it KEY}$_1$ is pressed. 

\item
Make a new Monitor Program project in the folder where you stored your source-code files.
In the Monitor Program screen illustrated in Figure~\ref{fig:exceptions}, make sure 
to choose {\sf Exceptions} in the {\it Linker Section Presets} drop-down menu.
Compile, download, and test your program. 
\end{enumerate}

\begin{figure}[H]
\begin{center}
\lstinputlisting[style=defaultNiosStyle, caption={Main program for Part 1.}, captionpos=b, label={fig:code}]{../design_files/part1.s}
\end{center}
\caption{Main program for Part 1.}
\label{fig:code}
\end{figure}

\begin{figure}[H]
\begin{center}
\lstinputlisting[style=defaultNiosStyle, lastline=26]{../design_files/exceptions.s}
Figure~\ref{fig:handlers}: Exception handlers (Part \textit{a}).
\end{center}
\end{figure}

\begin{figure}[H]
\begin{center}
\lstinputlisting[style=defaultNiosStyle, firstline=27]{../design_files/exceptions.s}
\end{center}
\caption{Exception handlers (Part \textit{b}).}
\label{fig:handlers}
\end{figure}

\begin{figure}[h]
	\begin{center}
	\includegraphics[scale=0.58]{figures/exceptions.png}
	\end{center}
	\vspace{-0.25cm}\caption{Selecting the {\sf Exceptions} linker section.}
\label{fig:exceptions}
\end{figure}

\noindent
{\bf Part II}
~\\
~\\
\noindent
Consider the main program shown in Figure~\ref{fig:code2}. The code is required to set up 
the Nios II stack pointers and then enable interrupts. The main program calls the
subroutines {\it CONFIG\_TIMER} and {\it CONFIG\_KEYS} to set up the two ports. You
are to write each of these subroutines. Set up the Interval Timer to generate one interrupt
every 0.25 seconds.

~\\
\noindent
In Figure~\ref{fig:code2} the main program executes an endless loop writing the value of
the global variable {\it COUNT} to the green lights LED.  In the interrupt service routine for 
Interval Timer you are to increment the variable {\it COUNT} by the value of 
the {\it RUN} global variable, 
which should be either 1 or 0.  You are to toggle the value of the {\it RUN} global variable 
in the interrupt service routine for the pushbutton KEYs, each time a KEY is pressed.
When {\it RUN} = 0, the main program will display a static count on the green lights,
and when {\it RUN} = 1, the count shown on the green lights will increment every 0.25 seconds.

~\\
\noindent
Make a new Monitor Program project for this part, and assemble, download, and test your
code.


\begin{center}
\lstinputlisting[style=defaultNiosStyle, caption={Main program for Part 2.}, captionpos=b, label={fig:code2}]{../design_files/part2.s}
\end{center}


\noindent
{\bf Part III}
~\\
~\\
\noindent
Modify your program from Part II so that you can vary the speed at which the counter
displayed on the green lights is incremented. All of your changes for this part should be made
in the interrupt service routine for the pushbutton KEYs. The main program and the rest of
your code should not be changed.

~\\
\noindent
Implement the following behavior. When {\it KEY}$_0$ is pressed, the value of the {\it RUN}
variable should be toggled, as in Part II. Hence, pressing {\it KEY}$_0$ stops/runs
the incrementing of the {\it COUNT} variable.
When {\it SW}$_0$ is high and {\it KEY}$_1$ is pressed, 
the rate at which {\it COUNT} is incremented should be doubled, and when {\it SW}$_0$ is 
low and {\it KEY}$_1$ is pressed the rate should be halved. You should implement this feature 
by stopping the Interval Timer within the pushbutton KEYs interrupt service routine, modifying the load value used in the
timer, and then restarting the timer.

~\\
\noindent
\noindent
{\bf Part IV}
~\\
~\\
\noindent
For this part you are to display the current count as a clock.
Set up the timer to provide one interrupt each second. Use this
timer to increment a global variable called {\it COUNT}. You should use the {\it COUNT} variable as
a real-time clock that is shown on the Monitor Program Terminal window. Use the format 
\blue{MM}:\blue{SS}, where \blue{\it MM} are minutes and \blue{\it SS} are seconds.
You should be able to stop/run the clock by pressing {\it KEY}$_1$.
When the clock reaches \blue{59}:\blue{59}, it should wrap around to \blue{00}:\blue{00}.

~\\
\noindent
Make a new folder to hold your Monitor Program project for this part. To show the {\it TIME} variable in
the real-time clock format \blue{MM}:\blue{SS}, you can use the approach that was
followed for Part IV of Lab Exercise 4. In Lab Exercise 4 you used polled I/O with
the interval timer, whereas now you are using interrupts. The interrupt service routine for
the timer should display the real-time clock on the Terminal window.

\noindent
Make a new Monitor Program project and test your program. In the screen shown in Figure
\ref{fig:terminal}, make sure to select {\sf JTAG\_UART} as the {\it Terminal
device}. Without this setting no character output will appear on the Terminal window when 
your code writes to the JTAG* UART.

\begin{figure}[htb]
	\begin{center}
	\includegraphics[scale=0.58]{figures/terminal.png}
	\end{center}
	\vspace{-0.25cm}\caption{Specifying the {\it Terminal device}.}
\label{fig:terminal}
\end{figure}

~\\
\noindent
As a final exercise, add to your program the ability to slow down/speed up the  
timer, in the same way that you implemented this capability for the Interval Timer in Part III
of this exercise. Observe the behavior of the Terminal window as it displays the real-time 
clock value at various timer rates.  Discuss any anomolous behavior that you observe.


\input{\CommonDocsPath/copyright.tex}
\end{document}
