\documentclass[epsfig,10pt,fullpage]{article}

\newcommand{\LabNum}{6}
\newcommand{\CommonDocsPath}{../../../../common/docs}
\input{\CommonDocsPath/preamble.tex}

\begin{document}

\centerline{\huge Computer Organization}
~\\
\centerline{\huge Laboratory Exercise \LabNum}
~\\
\centerline{\large Using C code with the Nios\textsuperscript{\textregistered} II Processor}
~\\

\noindent
This is an exercise in using C code with the Nios\textsuperscript{\textregistered} II processor.  We will use 
the {\it \productNameMedTM{}} software to compile, load, and run application programs
written in the C language.  In this exercise you have to be familiar with both the C language 
and the Nios II assembly language.  You should read the parts of the Monitor Program tutorial that 
discuss the use of C code.  This tutorial can be accessed from Intel's FPGA University Program 
website, or by selecting {\sf Help $>$ Tutorial} within the Monitor Program software.
You also need to be familiar with a number of I/O ports in the DE0-Nano-SoC, DE0-Nano, or DE10-Nano Computers (depending on which board you are using), 
including the parallel ports connected to the green LEDs, and pushbutton switches, as well as 
the Interval Timer port. These I/O ports are described in the documentation for the Computer Systems. 

~\\
~\\
\noindent
{\bf Part I}

~\\
\noindent
In Exercise 1, Part II, you were given a program in the Nios II assembly language that finds the 
largest number in a list of 32-bit integers that is stored in the memory.  This code is 
reproduced in Figure~\ref{fig:code}. For this exercise you are to write a C-language program 
that implements this task.  Perform the following steps.

\begin{enumerate}
\item
Write your C code in a file called {\it part1.c}.  You should use the green {\it LED} lights
to display the result produced by the program. The parallel port in the computer system connected to
the green lights is memory-mapped at the address 0xFF200000, as illustrated in Figure~\ref{fig:LED}.

\begin{lstlisting}[language=C]
#include<stdio.h>
\end{lstlisting}

To include a list of data words in the C program, you can declare them as an array using
a statement such as

\begin{lstlisting}[language=C]
int LIST[8] = {7, 4, 5, 3, 6, 1, 8, 2}; // number of elements, element 1,
                                        // element 2, ...
\end{lstlisting}

\item
Make a new Monitor Program project for 
this part of the exercise. In the Monitor Program screen shown in
Figure~\ref{fig:MPprogtype} select {\sf C Program} in the {\it Program Type}
dropdown menu, and on the screen that follows select your {\it part1.c} file. In the screen of 
Figure~\ref{fig:MPterminal} set the {\it Terminal device} to {\sf JTAG UART}.
This setting causes the output to appear in the {\it Terminal} window of the Monitor Program 
graphical user interface.

~\\
\noindent
Compile and download your program. Examine the disassembled code and compare it
to the code shown in Figure~\ref{fig:code}. To see the assembly code corresponding to your 
C source code, use the {\bf Goto instruction} dialog box in the Monitor Program's Disassembly
window. As illustrated in Figure~\ref{fig:MPgoto}, type {\sf main} in the dialog box and then 
click on the {\sf Go} button to display your code. When you run the program, the result
produced by the program should be displayed in binary format on the green {\it LED} lights.

\end{enumerate}

\begin{figure}[H]
\begin{center}
\lstinputlisting[style=defaultNiosStyle]{../design_files/part1.s}
\end{center}
\caption{Assembly-language program that finds the largest number.}
\label{fig:code}
\end{figure}

\begin{figure}[H]
	\begin{center}
	\includegraphics[scale=0.58]{figures/figureMP_progtype.png}
	\end{center}
	\vspace{-0.25cm}\caption{Setting the program type.}
\label{fig:MPprogtype}
\end{figure}

\begin{figure}[H]
	\begin{center}
	\includegraphics[scale=0.58]{figures/figureMP_terminal.png}
	\end{center}
	\vspace{-0.25cm}\caption{Configuring the {\it Terminal} window.}
\label{fig:MPterminal}
\end{figure}


\begin{figure}[H]
	\begin{center}
	\includegraphics[scale=0.58]{figures/figureMP_goto.png}
	\end{center}
	\caption{Displaying the code for the C program.}
\label{fig:MPgoto}
\end{figure}

~\\
\noindent
{\bf Part II}
~\\
~\\
\noindent
On the DE0-Nano, DE0-Nano-SoC, and DE10-Nano Computers, the Nios II processor does not have access to enough memory to 
use library functions like {\it printf} (in the {\it stdio.h} library). In this section you will 
write C language functions to display the result of your program written in Part I on the JTAG* 
terminal window. You should write functions to:

~\\
\begin{itemize}
\item Print a single ASCII character to the JTAG UART terminal
\item Print a null terminated string ({\it char}* or {\it char}{[}{]})
\item Convert an integer into a string
\item Print an integer to the JTAG UART terminal
\end{itemize}
~\\
\noindent
Theses functions should be similar to the subroutines used to display the {\it COUNT } variable 
in Exercise 4, Part IV. Modify your program from Part I to print a message to the JTAG UART terminal 
with the results, instead of using the green lights {\it LED}.

\begin{figure}[H]
	\begin{center}
	\includegraphics[scale=1]{figures/figureLED.pdf}
	\end{center}
	\caption{The parallel port connected to the green lights {\it LED} on the computer system.}
\label{fig:LED}
\end{figure}

~\\
\noindent
{\bf Part III}
~\\
~\\
\noindent
In Exercise 2 you were given a program that uses shift and AND operations to find the
longest string of 1's in a word of data. The program is reproduced in
Figure~\ref{fig:shiftAND}.  In Parts III and IV of Exercise 2 you were asked to extend
this program so that it processed a list of data words, rather than just one word. Also,
the program was extended to compute the longest
strings of 1's, the longest string of 0's, and the longest string of alternating 1's and 0's
for any of the words in the list. The results of these computations were to be shown 
on the Terminal window of the Monitor Program. For this part of the exercise, you are 
to write a C-language program to implement these tasks.

\begin{figure}[H]
\begin{center}
\lstinputlisting[style=defaultNiosStyle]{../design_files/part3.s}
\end{center}
\caption{Assembly-language program that counts consecutive ones.}
\label{fig:shiftAND}
\end{figure}

~\\
\noindent
To include the list of data words in your C program, you can declare them as an array using
a statement such as

\noindent
\begin{lstlisting}[language=C]
int TEST_NUM[] = {0x0000e000, 0x3fabedef, 0x00000001, 0x00000002, 0x75a5a5a5, 
                  0x01ffC000, 0x03ffC000, 0x55555555, 0x77777777, 0x08888888, 
                  0x00000000};
\end{lstlisting}


\noindent
Display the count for the longest string of 1's, 0's, and alternating 1's and 0's
on the Terminal window.

~\\
\noindent
Create a new folder and Monitor Program project for your C program, and then compile,
download, and test the code. Using the ten words of test data shown above, the correct
result that should appear on the Terminal window is:

\begin{center}
Longest string of ones: 12\\
Longest string of zeros: 31\\
Longest string of alternating ones/zeros: 32\\
\end{center}

~\\
\noindent
{\bf Part IV}
~\\
~\\
\noindent
In this part of the exercise you are to write a C program that implements a real-time
clock. Display the clock-time on the Terminal window in the format \blue{MM}:\blue{SS}:\blue{DD}, 
where \blue{\it MM} are minutes, \blue{\it SS} are seconds, and \blue{\it DD} are hundredths of 
a second. Pressing pushbutton {\it KEY}$_0$ should reset the time and display 
\blue{00}:\blue{00}:\blue{00} on the next line of the Terminal window. Since you cannot update 
the Terminal window at the rate of 1/100 seconds (the communication speed with the Terminal 
is too slow), you should display the current time, on the next line of the Terminal, only when
pushbutton {\it KEY}$_1$ is pressed. Measure time intervals of 0.01 seconds in your program 
by using polled I/O with the Interval Timer (a similar exercise was described in Part IV of Exercise 4, 
but using time intervals of 1 second). When the clock reaches \blue{59}:\blue{59}:\blue{99} it should wrap around 
to \blue{00}:\blue{00}:\blue{00}.
~\\
~\\
\noindent
Make a new folder to hold your Monitor Program project for this part. Create a
file called {\it part4.c} and type your C code into this file.  Make a new Monitor Program 
project for this part of the exercise, and then compile, download, and test your program. 

~\\
~\\
~\\
\noindent
{\bf Part V}
~\\
~\\
\noindent
Write a C program that scrolls the message \blue{\sf DE0-Nano-SoC} (or any message of your choice) 
in the right-to-left direction on the Terminal window. The contents of the display as the message scrolls
should appear as illustrated in Table 1. You should scroll the display at a rate of 
0.25 seconds per character.  You should be able to stop/run the scrolling message by pressing 
any pushbutton {\it KEY}.
It is not necessary to scroll the message all the way from the right-hand side of the
Terminal window; just start the message far enough to the right so that you can achieve 
the scrolling-effect illustrated in Table 1.
~\\
~\\
\begin{minipage}[t]{12.5 cm}
\begin{center}
\begin{tabular}{c|cccccccccccc}
Time slot & \multicolumn{6}{c}{Display} \\
\hline
{\rule[0mm]{0mm}{5mm}0} & & & & & & & & & & & &\blue{D}\\
1 & & & & & & & & & & & \blue{D} & \blue{E}\\
2 & & & & & & & & & & \blue{D} & \blue{E} & \blue{0}\\
3 & & & & & & & & & \blue{D} & \blue{E} & \blue{0} & \blue{-}\\
4 & & & & & & & & \blue{D} & \blue{E} & \blue{0} & \blue{-} & \blue{N}\\
5 & & & & & & & \blue{D} & \blue{E} & \blue{0} & \blue{-} & \blue{N} & \blue{a}\\
6 & & & & & & \blue{D} & \blue{E} & \blue{0} & \blue{-} & \blue{N} & \blue{a} & \blue{n}\\
7 & & & & & \blue{D} & \blue{E} & \blue{0} & \blue{-} & \blue{N} & \blue{a} & \blue{n} & \blue{o}\\
8 & & & & \blue{D} & \blue{E} & \blue{0} & \blue{-} & \blue{N} & \blue{a} & \blue{n} & \blue{o} & \blue{-}\\
9 & & & \blue{D} & \blue{E} & \blue{0} & \blue{-} & \blue{N} & \blue{a} & \blue{n} & \blue{o} & \blue{-} & \blue{S}\\
10 & & \blue{D} & \blue{E} & \blue{0} & \blue{-} & \blue{N} & \blue{a} & \blue{n} & \blue{o} & \blue{-} & \blue{S} & \blue{o}\\
11 & \blue{D} & \blue{E} & \blue{0} & \blue{-} & \blue{N} & \blue{a} & \blue{n} & \blue{o} & \blue{-} & \blue{S} & \blue{o} & \blue{C}\\
12 & \blue{E} & \blue{0} & \blue{-} & \blue{N} & \blue{a} & \blue{n} & \blue{o} & \blue{-} & \blue{S} & \blue{o} & \blue{C} & \\
13 & \blue{0} & \blue{-} & \blue{N} & \blue{a} & \blue{n} & \blue{o} & \blue{-} & \blue{S} & \blue{o} & \blue{C} & & \\
14 & \blue{-} & \blue{N} & \blue{a} & \blue{n} & \blue{o} & \blue{-} & \blue{S} & \blue{o} & \blue{C} & & & \\
15 & \blue{N} & \blue{a} & \blue{n} & \blue{o} & \blue{-} & \blue{S} & \blue{o} & \blue{C} & & & & \\
$\ldots$ & . & . & . &\\
\end{tabular}
\end{center}


\begin{center}
		  Table 1. Scrolling the message \blue{\sf DE0-Nano-SoC} on the Terminal window.
\end{center}
\end{minipage}

~\\
~\\
\noindent
You may want to make use of the {\it video-terminal} command shown below.
Sending this string of characters to the Terminal window causes it to return the ``cursor'' to 
the top-left corner of the window. The command can be declared as a string, named {\it home} in
this example, that can be sent to the Terminal window:

\begin{lstlisting}[language=C]
char home[] = "\033[H";
\end{lstlisting}

~\\
Note that scrolling a message across the Terminal window is similar in nature to the 
task of implementing a real-time clock, from Part IV. You should be able to reuse most of 
your code from Part IV. But instead of updating the clock display each time the A9 Private Timer
expires, you need to update the scrolling message.  

~\\
\noindent
Make a new folder to hold your Monitor Program project for this part. Create a
file called {\it part5.c} and type your C code into this file.  Make a new Monitor Program 
project, compile, download, and test your program. 




\input{\CommonDocsPath/copyright.tex}
\end{document}
