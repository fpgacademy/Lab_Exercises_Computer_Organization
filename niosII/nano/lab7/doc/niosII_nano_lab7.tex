\documentclass[epsfig,10pt,fullpage]{article}

\newcommand{\LabNum}{7}
\newcommand{\CommonDocsPath}{../../../../common/docs}
\input{\CommonDocsPath/preamble.tex}

\begin{document}

\centerline{\huge Computer Organization}
~\\
\centerline{\huge Laboratory Exercise \LabNum}
~\\
\centerline{\large Using Interrupts with C code}
~\\


The purpose of this exercise is to investigate the use of interrupts for the Nios\textsuperscript{\textregistered} II
processor, using C code. To do this exercise you need to have a good working knowledge of
the exceptions processing mechanisms of the Nios~II processor.  You should also be
familiar with the parts of the DE0-Nano-SoC, DE0-Nano, or DE10-Nano Computer documentation that pertain 
to the use of exceptions and interrupts with C code.

~\\
This exercise involves the same tasks as those given in Exercise 5,
except that this exercise uses C code rather than assembly-language code.

\section*{Part I}
\addcontentsline{toc}{1}{Part I}
Consider the main program shown in Figure~\ref{fig:code}. The program calls a 
subroutine {\it config\_KEYs()} to initialize the pushbutton KEYs port so that it will generate
interrupts, and calls a subroutine {\it enable\_nios2\_interrupts()} to enable
interrupts in the Nios~II processor. You are to fill in the missing code for the 
subroutines. To enable interrupts the main program includes
{\it macros}, in the file "nios2\_ctrl\_reg\_macros.h", which provide access to the Nios~II
status and control registers.  Examples of useful macros that might be included 
are provided in Figure \ref{fig:macros}. 

~\\
After completing the initialization steps
described above, the main program just ``idles'' in an endless loop.
The purpose of the program is to toggle the state of the {\it LED}0 and {\it LED}1, when a corresponding 
pushbutton {\it KEY} is pressed. Since the main program only idles in a loop, the displays 
have to be controlled by using an interrupt service routine for the pushbutton KEYs port. 

\begin{enumerate}
\item Create a new folder to hold your Monitor Program project for this part. Create a
file, such as {\it part1.c}, for your main program, and create any other source-code files 
that you may wish to use.  Write the code for the subroutines that are called by the 
main program. Be sure to enable Nios~II interrupts for the pushbutton KEYs port.

\item 
The reset and exception handlers for the main program are given in Figure \ref{fig:irq_code}. 
The function called {\it the\_reset} provides a simple reset mechanism by
performing a branch to the main program. The function named {\it the\_exception} 
represents a general exception handler that can be used with any C program. It includes 
assembly language code to check if the exception is caused by an external interrupt, and, 
if so, calls a C language routine named {\it interrupt\_handler}. This routine can then 
perform whatever action is needed for the specific application. In 
Figure~\ref{fig:irq_code}, the {\it interrupt\_handler} code first 
determines which exception has occurred, by using a macro from Figure~\ref{fig:macros} 
that reads the content of the Nios II interrupt pending register. 
  
\newpage
You have to write the code for the {\it pushbutton\_isr()} interrupt service routine.
Your code should light up {\it LED}1 when {\it KEY}$_1$ is pressed, and then if {\it KEY}$_1$
is pressed again, {\it LED}1 should be turned off. You should toggle the {\it LED}1 display between
on and off in this manner each time. The same should be done 
for {\it KEY}$_0$ and {\it LED}0.

\item
Make a new Monitor Program project in the folder where you stored your source-code files.
In the Monitor Program screen illustrated in Figure~\ref{fig:exceptions}, make sure 
to choose {\sf Exceptions} in the {\it Linker Section Presets} drop-down menu.

\item
Compile, download, and test your program. 
\end{enumerate}

\begin{figure}[H]
\begin{center}
\lstinputlisting[language=C]{../design_files/part1.c}
\end{center}
\caption{Main program for Part I.}
\label{fig:code}
\end{figure}

\newpage
\begin{figure}[h!]
\begin{center}
\lstinputlisting[language=C]{../design_files/nios2_ctrl_reg_macros.h}
\end{center}
	\vspace{-0.5cm}\caption{Macros for accessing Nios II status and control registers.}
   \label{fig:macros}
\end{figure}

\newpage

\lstinputlisting[language=C, linerange=1-23, label={fig:irq_code}]{../design_files/exception_handler.c}
\lstinputlisting[language=C, linerange=24-69]{../design_files/exception_handler.c}

\begin{figure}[H]
\begin{center}
\lstinputlisting[language=C, linerange=70-89]{../design_files/exception_handler.c}
\end{center}
\vspace{-0.5cm}\caption{Reset and exception handler C code.}
\end{figure}

\begin{figure}[H]
	\begin{center}
	\includegraphics[scale=0.58]{figures/exceptions.png}
	\end{center}
	\vspace{-0.25cm}\caption{Selecting the {\sf Exceptions} linker section.}
\label{fig:exceptions}
\end{figure}

\section*{ Part II}
\addcontentsline{toc}{2}{Part II}
Consider the main program shown in Figure~\ref{fig:code2}. The code is required to set up 
interrupts from two sources: the Interval Timer and the pushbutton KEYs port. The main program 
calls the subroutines {\it config\_timer( )} and {\it config\_KEYS( )} to set up the two ports. 
You are to write each of these subroutines. Set up the Interval Timer to generate one interrupt
every 0.25 seconds.

~\\
In Figure~\ref{fig:code2} the main program executes an endless loop writing the value of
the global variable {\it count} to the green lights LED.  In the interrupt service routine for 
the Interval Timer you are to increment the variable {\it count} by the value of the {\it run} global variable, 
which should be either 1 or 0.  You are to toggle the value of the {\it run} global variable 
in the interrupt service routine for the pushbutton KEYs, each time a KEY is pressed.
When {\it run} = 0, the main program will display a static count on the red lights,
and when {\it run} = 1, the count shown on the red lights will increment every 0.25 seconds.
Make a new Monitor Program project for this part, and assemble, download, and test your
code.

\begin{figure}[H]
\begin{center}
\lstinputlisting[language=C]{../design_files/part2.c}
\end{center}
\vspace{-0.5cm}\caption{Main program for Part II.}
\label{fig:code2}
\end{figure}

\section*{ Part III}
\addcontentsline{toc}{3}{Part III}
Modify your program from Part II so that you can vary the speed at which the counter
displayed on the green lights is incremented. All of your changes for this part should be made
in the interrupt service routine for the pushbutton KEYs. The main program and the rest of
your code should not be changed.

~\\
Implement the following behavior. When {\it KEY}$_1$ is pressed, check the status of switches 0
and 1 (SW0 and SW1). If SW1 is high (a value of 1), you should toggle the {\it run} variable, 
similar to Part II. If SW1 is low (a value of 0), you should check the status of SW0. If SW0
is high, the rate at which the {\it count} variable is incremented should be doubled, and if SW0
is low, the rate should be halved. You should implement this feature by stopping the Interval Timer within
the pushbutton KEYs interrupt service routine, modifying the load value used in the
timer, and then restarting the timer.

\newpage
\section*{ Part IV}
\addcontentsline{toc}{4}{Part IV}
For this part you are to create a real-time clock that is shown on the JTAG* UART terminal window.
Set up an interval timer to provide an interrupt every 1/100 of a second. Use this
timer to increment a global variable called {\it time}. You should use the {\it time} variable as your real time clock.
Use the format \red{MM}:\red{SS}:\red{DD}, where \red{\it MM} are minutes, \red{\it SS} are seconds and \red{\it DD} are hundredths of a second.
When the clock reaches \red{59}:\red{59}:\red{99}, it should wrap around to \red{00}:\red{00}. 
You should be able to reuse the functions written in Example 6 to display 
characters, strings and numbers on the JTAG UART terminal (remember the Nios II processor(s) on the DE0-Nano, 
DE0-Nano-SoC, and DE10-Nano boards do not have access to enough memory for library functions like {\it printf}).
Since you cannot update the Terminal window at the rate of 1/100 seconds 
(the communication speed with the Terminal is too slow), you should display the current time, 
only when a pushbutton {\it KEY} is pressed.

~\\
Make a new folder to hold your Monitor Program project for this part. Write the program for the
real-time clock. To show the {\it TIME} variable in the real-time clock format
\red{MM}:\red{SS}, you can use the same approach that was followed for Part 4 of Lab Exercise 4.
In that previous exercise you used polled I/O with the Interval Timer,
whereas now you are using interrupts. One possible way to structure your code is illustrated in
Figure~\ref{fig:code3}. 

~\\
Using the scheme in Figure~\ref{fig:code3}, the interrupt service routine for  
Interval Timer has to increment the {\it TIME} variable.

~\\
Make a new Monitor Program project and test your code. 

\begin{figure}[H]
\begin{center}
\lstinputlisting[language=C]{../design_files/part4.c}
\end{center}
\vspace{-0.5cm}\caption{Main program for Part IV.}
\label{fig:code3}
\end{figure}



\input{\CommonDocsPath/copyright.tex}
\end{document}
