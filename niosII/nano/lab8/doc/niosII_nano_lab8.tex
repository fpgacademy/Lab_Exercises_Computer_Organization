\documentclass[epsfig,10pt,fullpage]{article}

\newcommand{\LabNum}{8}
\newcommand{\CommonDocsPath}{../../../../common/docs}
\input{\CommonDocsPath/preamble.tex}

\begin{document}

\centerline{\huge Computer Organization}
~\\
\centerline{\huge Laboratory Exercise \LabNum}
~\\
\centerline{\large Using Interrupts with Assembly Language and C code}
~\\

\noindent
The purpose of this exercise is to investigate the use of interrupts for the Nios\textsuperscript{\textregistered} II
processor, using both assembly language and C code. To do this exercise you need to be familiar 
with the exceptions processing mechanisms of the Nios II processor.
You should also read the parts of the DE0-Nano, DE0-Nano-SoC, or DE10-Nano Computer documentation that pertain 
to the use of exceptions and interrupts.

~\\
\noindent
{\bf Part I}
~\\\\
\noindent
The main program in Figure~\ref{fig:code} consists of an endless loop that comprises 
a few logical instructions. The code in this loop does not serve any useful purpose other 
than to provide a specific set of instructions that are being executed repeatedly: logical
AND, Exclusive-OR, logical OR, and branch. When a pushbutton KEY is pressed to cause an 
interrupt, the main program may be executing any of the instructions in the loop. After 
this instruction is completed, the processor will branch to the interrupt handler. 
Figure~\ref{fig:handler} shows most of the code for the interrupt handler. It passes to 
the KEYs interrupt service routine, as a parameter in register r4, the machine code 
of the next instruction to be executed in the main program. You have to write the KEYs interrupt 
service routine. It should display on the Terminal window of the Monitor Program which 
instruction will be executed when the processor returns to the main program. 
You should use many of the concepts and subroutines from Exercise 7.
An example of output on the Terminal window after several KEY presses might be:

\begin{minipage}[t]{12.5 cm}
\begin{lstlisting}
and
or
br
xor
xor
and
...
\end{lstlisting}
\end{minipage}

~\\
\noindent
Perform the following:

\begin{enumerate}
\item Create a new folder to hold your Monitor Program project for this part. Create a
file, such as {\it part1.s}, and type the assembly language code for the main program 
into this file. 

\item Create any other source code files you may want, and write the code for the
{\it EXCEPTION\_HANDLER} and the subroutine to handle interrupts from a pushbutton key. 
Set the exception handler to send interrupts to the Nios II processor from the pushbutton KEYs port. 

\begin{figure}[H]
\begin{center}
\lstinputlisting[style=defaultNiosStyle]{../design_files/part1.s}
\end{center}
\caption{Main program for Part I.}
\label{fig:code}
\end{figure}


\begin{figure}[H]
\begin{center}
\lstinputlisting[style=defaultNiosStyle]{../design_files/exception_handler.s}
\end{center}
\caption{Exception handlers.}
\label{fig:handler}
\end{figure}
~\\

\item 
To implement the KEYs interrupt service routine, you need to know the Nios II
instruction-encodings that are shown in Figure~\ref{fig:encoding}.  Part $a$ of the figure 
gives the encoding of R-Type instructions, which include and, xor, and 
or. Figure~\ref{fig:encoding}$b$ gives the encoding of I-Type instructions, which includes
the br instruction. 

In part $a$ of Figure~\ref{fig:encoding} the A, B and C fields represent the three registers
in the operation. The C field is the destination register, and the A and B fields are the operation registers.
For example, an {\it and} operation would look like: { \sf and rC, rA, rB }.
For the R-type instructions {\sf and, or, xor}, the OP field will always have the value 0x3A.
The 11-bit {\sf OPX} field can be used to differentiate between processing instructions: it has values 
{\sf 0x1C0} for {\bf and}, {\sf 0x3C0} for {\bf xor} and {\sf 0x2C0} for {\bf or}. The {\sf br} instruction
is an I-type instruction, which has the format shown in Figure~\ref{fig:encoding}$b$. The IMM16
field specifies a 16-bit immediate value. The OP field holds the operation to be performed. The operation code 
for the {\bf br} instruction is {\sf 0x06}. For the branch instruction, the A and B field can be ignored.
For more information on OP codes and Nios II instructions, please see the {\it Nios II Instruction Set} document.
\\\\
{\it Hint: Research the difference between the {\sf ra} and {\sf ea} registers on the Nios II Processor.}

~\\
\begin{figure}[htb]
	\begin{center}
	\includegraphics[width=5.5in]{figures/encodingDataProcess.png}

	~\\
	{\sf (a) Nios II R-Type Instruction format}

	~\\
	\includegraphics[width=5.5in]{figures/encodingB.png}

	~\\
	{\sf (b) Nios II I-Type Instruction format}
	\end{center}
	\vspace{-0.25cm}\caption{The machine code format.}
\label{fig:encoding}
\end{figure}

\newpage
To display the name of an instruction mnemonic on the Terminal window you may wish to 
declare some strings of data as indicated below, and write the characters in these strings to the 
JTAG* UART that is connected to the Terminal window:

\begin{minipage}[t]{12.5 cm}
\begin{lstlisting}[style=defaultNiosStyle]
/* .skip is for padding word alignment*/
AND: 		.string "and\n"
			.skip 3
OR: 		.string "or\n"
XOR: 		.string "xor\n"
			.skip 3
B: 			.string "br\n"
\end{lstlisting}
\end{minipage}

\item
Make a new Monitor Program project in the folder where you stored your source-code files.
In the Monitor Program screen illustrated in Figure~\ref{fig:exceptions}, 
choose {\sf Exceptions} in the {\it Linker Section Presets} drop-down menu.
In the screen of Figure~\ref{fig:terminal}, select {\sf JTAG\_UART} as 
the {\it Terminal device}.  Refer to Exercise 2, Part IV, for information on using 
the JTAG UART to communicate with the Monitor Program's Terminal window.  

\item
Compile, download, and test your program.
\end{enumerate}

~\\
\begin{figure}[H]
	\begin{center}
	\includegraphics[scale=0.58]{figures/exceptions_s.png}
	\end{center}
	\vspace{-0.25cm}\caption{Selecting the {\sf Exceptions} linker section.}
\label{fig:exceptions}
\end{figure}
\newpage

\begin{figure}[htb]
	\begin{center}
	\includegraphics[scale=0.58]{figures/terminal.png}
	\end{center}
	\vspace{-0.25cm}\caption{Specifying the {\it Terminal device}.}
\label{fig:terminal}
\end{figure}

\noindent
{\bf Part II}
~\\
~\\
For this part you are to modify your code from Part I so that additional information in
displayed on the Terminal window when a pushbutton KEY is pressed. All of your changes for
this part should be done in the interrupt service routine for the pushbutton KEYs. You
should not have to change any other files. For Part I you were required to display only 
the mnemonic of the next instruction, such as {\sf and} or {\sf xor}, that is to be executed in the
main program on return from an interrupt. For this part you are to extend your solution
so that it also displays the names of the register arguments of each instruction. Use the
machine encodings shown in Figure~\ref{fig:encoding} to find the required information. An 
example of output produced on the Terminal window after several KEYs have been pressed
might be:

\begin{minipage}[t]{12.5 cm}
\begin{lstlisting}
or    r12, r11, r10
xor   r13, r14, r15
and   r10, r11, r12
or    r16, r17, r18
and   r18, r17, r16
...
\end{lstlisting}
\end{minipage}

\newpage
\noindent
{\bf Part III}
~\\
~\\
\noindent
Consider the C code shown in Figure~\ref{fig:C_code1}. It contains an endless loop that
repeatedly calls a function named {\it KEY\_pressed} to see if a pushbutton KEY has been pressed.
The function {\it KEY\_pressed}, not shown in the figure, checks to see if a KEY is
currently being pressed. If so, it waits for the KEY to be released and then returns 1. 
Otherwise, if no KEY was pressed, 0 is returned. When {\it KEY\_pressed} returns 1, the main
program calls a function named {\it doit}. This function is supposed to print, on the Terminal 
window, the mnemonic of the instruction that will be executed on return from the {\it doit}
function. In the main program {\it inline assembly} commands are used to insert specific
instructions following each call to {\it doit}. For example, the first call to {\it doit}
is followed by the instruction {\sf and r10, r11, r12}, the second call to {\it doit} is
followed by the instruction {\sf xor r13, r14, r15}, and so on.

~\\
\noindent
Part of the code for the {\it doit} function is given in Figure\ref{fig:C_code1}. It uses two
inline assembly commands. The command


\begin{center}
\begin{tabular}{c}
\begin{lstlisting}[language=C]
asm("ldw r10, 0(ra)" : : : "r10")
\end{lstlisting}
\end{tabular}
\end{center}

~\\
\noindent
loads the machine code of the instruction pointed to by the return address register into register r10. The 
argument "r10" in this command informs the C compiler that the contents of register r10 will be 
changed as a result of executing the instruction. The command

\begin{center}
\begin{tabular}{c}
\begin{lstlisting}[language=C]
asm("mov \%0, r10" : "=r" (machine_code) : : )
\end{lstlisting}
\end{tabular}
\end{center}

~\\
\noindent
copies the contents of register r10 into the variable {\it machine\_code}. The argument {\sf \%0}
in this command is set to whichever register, for example register r11, is chosen by the C
compiler to hold the value of the variable {\it machine\_code}.  

~\\
\noindent
Make sure that you understand how the inline assembly commands work. Documentation on
these commands can be found by searching on the web for {\it gnu inline assembly Nios II}, or
something similar.

~\\
\noindent
Perform the following:

\begin{enumerate}
\item
Type the C code for the main program into a file, for example {\it part3.c}. Also, add
your own code for the {\it KEY\_pressed} function.

\item 
Complete the C code for the {\it doit} function. You should display on the Terminal window
the mnemonic of the instruction corresponding to the {\it machine\_code} variable.
Refer to Figure~\ref{fig:encoding} for the machine code formats.
Remember from previous exercises that the DE0-Nano, DE0-Nano-SoC, and DE10-Nano boards do not have access
to enough memory to use library functions like {\it printf}. You should use the functions you 
developed in previous exercises (Exercises 6 and 7) to print characters, strings and numbers
to the JTAG terminal window.
\\\\
{\it Hint: In this example we are not using interrupts, how does this change the use of the 
{\sf ra} and {\sf ea} registers in terms of accessing the next instruction to be executed in the main function?}

\item
Make a new Monitor Program project for this part of the exercise. In the Monitor Program screen 
shown in Figure~\ref{fig:MPterminal} set the {\it Terminal device} to {\sf JTAG\_UART}.

\item
Compile, download, and test your program. 
\end{enumerate}

\begin{figure}[H]
\begin{center}
\lstinputlisting[language=C]{../design_files/part3.c}
\end{center}
\caption{Main program for Part III.}
\label{fig:C_code1}
\end{figure}

\newpage
\begin{figure}[htb]
	\begin{center}
	\includegraphics[scale=0.58]{figures/figureMP_terminal.png}
	\end{center}
	\vspace{-0.25cm}\caption{Specifying the {\it Terminal device}.}
\label{fig:MPterminal}
\end{figure}

~\\
~\\
\noindent
{\bf Part IV}
~\\
~\\
\noindent
In this part you are to repeat the tasks given in Parts I and II, but using C code rather
than assembly-language code.

~\\
\noindent
Consider the main program shown in Figure~\ref{fig:C_code2}. The main program calls the subroutine 
{\it config\_KEYs()} to initialize the pushbutton KEYs port so that it will generate interrupts. 
Finally, a subroutine {\it enable\_nios2\_interrupts()} is called to unmask IRQ interrupts in the Nios II processor.

~\\
\noindent
After completing the initialization steps
described above, the main program executes an endless loop that consists of several
logical instructions. Inline assembly commands, described in Part III, are used to specify 
the logic instructions.

\begin{figure}[htb]
\begin{center}
\lstinputlisting[language=C]{../design_files/part4.c}
\end{center}
\caption{Main program for Part IV.}
\label{fig:C_code2}
\end{figure}

~\\
\noindent
Perform the following:

\begin{enumerate}
\item Create a new folder to hold your Monitor Program project for this part. Create a
file, such as {\it part1.c}, for your main program, and create any other source-code files 
that you may wish to use.  Write the code for the subroutines that are called by the 
main program.

\item 
Use the exception handler code you created for Exercise 7 which called the {\it interrupt\_handler},
and in turn called the {\it pushbutton\_ISR()} function. You should not need to change the exception 
handler or interrupt handler code you wrote, you should only need to change the functionality of the
{\it pushbutton\_ISR()} function.  
~\\
Your code should perform the same task as in Part II of this exercise. That is, you are to
print on the Terminal window the mnemonic, and register indices, of the instruction that 
will be executed on return from the interrupt. You can make the same assumptions as in
Parts I and II, namely that only the logical instructions, plus the branch instruction,
have to be handled by your code

\item
Make a new Monitor Program project in the folder where you stored your source-code files.
As discussed for Part III, in the Monitor Program screen shown in 
Figure~\ref{fig:MPterminal} set the {\it Terminal device} to {\sf JTAG\_UART}.
Also, in the Monitor Program screen illustrated in Figure~\ref{fig:exceptions}, make sure 
to choose {\sf Exceptions} in the {\it Linker Section Presets} drop-down menu.

\item
Compile, download, and test your program. 
\end{enumerate}

~\\
\begin{figure}[htb]
	\begin{center}
	\includegraphics[scale=0.58]{figures/exceptions_C.png}
	\end{center}
	\vspace{-0.25cm}\caption{Selecting the {\sf Exceptions} linker section.}
\label{fig:exceptions}
\end{figure}


\input{\CommonDocsPath/copyright.tex}
\end{document}
