\documentclass[epsfig,10pt,fullpage]{article}

\newcommand{\LabNum}{2}
\newcommand{\CommonDocsPath}{../../../../common/docs}
\input{\CommonDocsPath/preamble.tex}

\begin{document}

\centerline{\huge Computer Organization}
~\\
\centerline{\huge Laboratory Exercise \LabNum}
~\\
\centerline{\large Using Logic Instructions with the Nios\textsuperscript{\textregistered} II Processor}
~\\


Logic instructions are needed in many embedded applications.  Logic instructions are useful 
for manipulation of bit strings and for dealing with data at the bit level where only a 
few bits may be of special interest.  They are essential in dealing with input/output tasks.
In this exercise we will consider some typical uses.
We will use the Nios\textsuperscript{\textregistered} II processor in one of the pre-built computer systems that come with the Monitor Program.

\section*{Part I}
\addcontentsline{toc}{1}{Part I}
In this part you will implement a Nios II assembly language program that counts the longest 
string of 1's in a word of data. For example, if the word of data is {\sf 0x103fe00f}, then 
the required result is 9.

~\\
Perform the following:

\begin{enumerate}
\item Create a new folder to hold your Monitor Program project for this part. Create a
file called {\it part1.s}, and type the assembly language code shown in 
Figure~\ref{fig:code} into this file.
This code uses an algorithm involving shift and AND operations to find the required 
result---make sure that you understand how this works.

\item
Make a new Monitor Program project in the folder where you stored the {\it part1.s}
file. Depending on your DE-series board, use the corresponding computer system 
listed in Table~\ref{tab:computer_systems}.

\begin{table}[H]
	\begin{center}
	\begin{tabular}{ l | l }
	\bf{Board} & \bf{Computer System} \\
	\hline
	\rule{0pt}{3ex}DE0-CV & DE0-CV Computer \\ 
	DE1-SoC & DE1-SoC Computer \\
	DE2-115 & DE2-115 Computer \\
	DE10-Lite & DE10-Lite Computer \\
	DE10-Standard & DE10-Standard Computer \\
	\end{tabular}
	\caption{DE-series board computer systems}
	\label{tab:computer_systems}
	\end{center}
\end{table}

\item
Compile and load the program. Fix any errors that you encounter (if you mistyped some of
the code). Once the program is loaded into memory in the computer, single step
through the code to see how the program works.
\end{enumerate}

\newpage
\section*{Part II}
\addcontentsline{toc}{2}{Part II}
Perform the following:
\begin{enumerate}
\item Make a new folder and make a copy of the file {\it part1.s} in that new folder. Give
the new file a name such as {\it part2.s}.
\item In the new file {\it part2.s}, take the code which calculates the 
number of consecutive 1's and make it into a subroutine called ONES. Have the subroutine use 
register r4 to receive the input data and register r2 for returning the result.
\item Add more words in memory starting from the label TEST\_NUM. You can add as many
words as you like, but include at least 10 words. To terminate the list include the word 0
at the end---check for this 0 entry in your main program to determine when all of the
items in the list have been processed.
\item In your main program, call the newly-created subroutine in a loop for every word of 
data that you placed in memory. Keep track of the longest string of 1's in any of the words, 
and have this result in register r10 when your program completes execution. 
\item Make sure to use breakpoints or single-stepping in the Monitor Program to observe what 
happens each time the ONES subroutine is called.
\end{enumerate}

\begin{figure}[H]
\begin{center}
\lstinputlisting[style=defaultNiosStyle]{../design_files/part1.s}
\end{center}
\caption{Assembly-language program that counts consecutive ones.}
\label{fig:code}
\end{figure}

\newpage
\section*{Part III}
\addcontentsline{toc}{3}{Part III}
One might be interested in the longest string of 0's, or even the longest string of
alternating 1's and 0's. For example, the binary number {\sf 101101010001} has a string of 
6 alternating 1's and 0's.

~\\
 Write a new assembly language program that determines the following:
\begin{itemize}
\item Longest string of 1's in a word of data---put the result into register r10
\item Longest string of 0's in a word of data---put the result into register r11
\item Longest string of alternating 1's and 0's in a word of data---put the result into 
register r12
(Hint: What happens when an n-bit number is XORed with an n-bit string of alternating 0's and 1's?)
\end{itemize}
Make each calculation in a separate subroutine called ONES, ZEROS, and ALTERNATE. Call
each of these subroutines in the loop that you wrote in Part III, and keep track of the
largest result for each calculation, from your list of data.

\section*{Part IV}
\addcontentsline{toc}{4}{Part IV}
In this part you are to extend your code from Part III so that the results produced are
shown on the 7-segment displays on your DE-series board. Display the longest string of 1's
(r10) on {\it HEX}$1-0$, the longest string of 0's (r11) on {\it HEX}$3-2$, and 
the longest string of alternating 1's and 0's (r12) on {\it HEX}$5-4$.  

~\\
Each result should be displayed as a two-digit decimal number. You may want to use the approach
discussed in Part IV of Exercise 1 to convert the numbers in registers r10, r11, and r12 from
binary to decimal.  

~\\
The parallel ports in the computer systems that are connected to the 7-segment displays 
are memory mapped devices.
Figure~\ref{fig:HEX} shows how the display segments are connected to the parallel ports 
and the ports' respective memory mapped addresses.

~\\
To show each of the numbers from \red{0} to \red{9} it is necessary
to light up the appropriate display segments.
For example, to show \red{0} on {\it HEX}0 you have to turn on all 
of the segments except for the middle one (segment 6). Hence, you would store the
bit-pattern $(00111111)_2$ into the address corresponding to the {\it HEX}$3-0$ parallel port to show this result.
A subroutine that produces such bit patterns in given in Figure~\ref{fig:code7}.

\begin{figure}[H]
	\begin{center}
	\includegraphics[scale=.75]{figures/figureHEX.pdf}
	\end{center}
	\caption{The parallel ports connected to the seven-segment displays, {\it HEX}$5-0$, of the DE-series board.}
\label{fig:HEX}
\end{figure}

An example of code that shows the contents of registers on the 7-segment displays is
illustrated in Figure~\ref{fig:codefrag}. Note that this code uses the DIVIDE subroutine that 
was discussed in Part IV of Exercise 1.  The code in the figure shows only the steps needed 
for register r10.  Extend the code to display all three result registers on the 7-segment
displays as described above.

\begin{figure}[H]
\begin{center}
\lstinputlisting[style=defaultNiosStyle, linerange=1-13]{../design_files/part4.s}
\end{center}
\caption{A subroutine that produces bit patterns for 7-segment displays.}
\label{fig:code7}
\end{figure}

\begin{figure}[H]
\begin{center}
\lstinputlisting[style=defaultNiosStyle, firstline=16]{../design_files/part4.s}
\end{center}
\caption{A code fragment for showing registers in decimal on 7-segment displays.}
\label{fig:codefrag}
\end{figure}


\input{\CommonDocsPath/copyright.tex}
\end{document}
