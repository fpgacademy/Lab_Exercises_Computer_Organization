\documentclass[epsfig,10pt,fullpage]{article}

\newcommand{\LabNum}{4}
\newcommand{\CommonDocsPath}{../../../../common/docs}
\input{\CommonDocsPath/preamble.tex}

\begin{document}

\centerline{\huge Computer Organization}
~\\
\centerline{\huge Laboratory Exercise \LabNum}
~\\
\centerline{\large Input/Output in an Embedded System}
~\\

The purpose of this exercise is to investigate the use of devices that provide input and
output capabilities for a processor. There are two basic techniques for dealing with I/O devices:
program-controlled polling and interrupt-driven approaches.  You will use the polling approach 
in this exercise, writing programs in the Nios\textsuperscript{\textregistered} II assembly language.  Your programs will 
be executed on a Nios~II processor in one of the DE-series computer systems.
Parallel port interfaces, as well as a timer module, will be used as examples of I/O hardware.

~\\
A parallel port provides for data transfer in either the input or output direction.  The 
transfer of data is done in parallel and it may involve from 1 to 32 bits. The number of
bits, $n$, and the type of transfer depend on the specifications of the specific parallel port 
being used.  The parallel port interface can contain the four registers shown in
Figure~\ref{fig:parallel}.

\begin{figure}[H]
	\begin{center}
	\includegraphics[scale=1]{figures/figureparallel.pdf}
	\end{center}
	\caption{Registers in the parallel port interface.}
\label{fig:parallel}
\end{figure}


Each register is $n$ bits long. The registers have the following purpose:
\begin{itemize}
\item {\it Data} register: holds the $n$ bits of data that are transferred between the parallel 
port and the Nios~II processor. It can be implemented as an input, 
output, or a bidirectional register.  \item {\it Direction} register defines the direction 
 of transfer for each of the $n$ data bits when a bidirectional interface is generated.
\item {\it Interrupt-mask} register: used to enable interrupts from the
input lines connected to the parallel port.
\item {\it Edge-capture} register: indicates when a change of logic value is detected in 
the signals on the input lines connected to the parallel port. Once a bit in the edge
capture register becomes asserted, it will remain asserted. An edge-capture bit can be
de-asserted by writing to it using the Nios~II processor.
\end{itemize}

Not all of these registers are present in some parallel ports. For example,
the {\it Direction} register is included only when a bidirectional interface is specified.
The {\it Interrupt-mask} and {\it Edge-capture} registers must be included if
interrupt-driven input/output is used.

~\\
The parallel port registers are memory mapped, starting at a specific {\it base} address.
The base address has to be a multiple of four if the parallel port is to be accessed using
word accesses from the Nios~II processor. The base address becomes the address of the {\it Data} 
register in the parallel port. The addresses of the other three registers have offsets 
of 4, 8, or 12 bytes (1, 2, or 3 words) from this base address.
The pre-built computer system all have parallel ports connected to slider switches, pushbutton KEYs, 
LEDs, and seven-segment displays (when those peripherals exist).

\section*{Part I}
\addcontentsline{toc}{1}{Part I}
Write a Nios~II assembly-language program that displays a decimal digit on the seven-segment
display {\it HEX}0. The other seven-segment 
displays on your DE-series board should be blank.  

~\\
The parallel port connected to the 
seven-segment displays {\it HEX}$3-0$ is memory mapped at the address {\sf 0xFF200020}, 
and the port connected to {\it HEX}$5-4$ is at the address {\sf 0xFF200030}.
Figure~\ref{fig:HEX} shows how the display segments are connected to the parallel ports.  

\begin{figure}[H]
	\begin{center}
	\includegraphics[scale=1]{figures/figureHEX.pdf}
	\end{center}
	\caption{The parallel ports connected to the seven-segment displays, {\it HEX}$5-0$, in the DE-series copmuter systems.}
\label{fig:HEX}
\end{figure}

Initially the number displayed on {\it HEX}0 should be 0.  If {\it KEY}$_1$ is pressed then {\it SW}$_0$ 
should be checked. If {\it SW}$_0$ is high increment the displayed number, and 
if it is low then decrement the number. Pressing {\it KEY}$_0$ should 
blank the display, and pressing any other KEY after that should return the display to 0.
The parallel port connected to the pushbutton {\it KEYs} is illustrated in Figure~\ref{fig:KEY}. 
The parallel port connected to the slider switches {\it SWs} is illustrated in Figure~\ref{fig:SW}. 
In your program, use polled I/O 
to read the {\it Data} register to see when a button is being pressed. When you are not pressing 
any {\it KEY} the {\it Data} register provides~0, and when you press {\it KEY}$_i$ the 
{\it Data} register provides the value 1 in bit position $i$. Once a button-press is detected,
be sure that your program waits until the button is released. You should not use the 
{\it Interruptmask} or {\it Edgecapture} registers for this part of the exercise.

\begin{figure}[H]
	\begin{center}
	\includegraphics[scale=.9]{figures/figureKEY.pdf}
	\end{center}
	\caption{The parallel port connected to the pushbutton {\it KEYs}.}
\label{fig:KEY}
\end{figure}

\begin{figure}[H]
	\begin{center}
	\includegraphics[scale=.9]{figures/figureSW.pdf}
	\end{center}
	\caption{The parallel port connected to the slider switches {\it SWs}.}
\label{fig:SW}
\end{figure}

Perform the following:

\begin{enumerate}
\item Create a new folder to hold your Monitor Program project for this part. Create a
file called {\it part1.s} and type your assembly language code into this file.
You may want to refer to a discussion, and examples of assembly-language code,
in Part IV of Lab Exercise 2 about displaying numbers on seven-segment displays.

\item
Make a new Monitor Program project in the folder where you stored the {\it part1.s} file.
Select Nios~II as the target processor architecture and use the appropriate pre-built 
computer system for your DE-series board.

\item
Compile, download, and test your program. 
\end{enumerate}

\section*{Part II}
\addcontentsline{toc}{2}{Part II}
Write a Nios~II assembly-language program that displays a two-digit decimal counter on the 
seven-segment displays {\it HEX}$1-0$. The counter should be incremented approximately
every $0.25$ seconds. When the counter reaches the value 99, it should start again at 0.
The counter should stop/start when any pushbutton {\it KEY} is pressed.

~\\
To achieve a delay of approximately 0.25 seconds, use a delay-loop in your assembly language
code. A suitable example of such a loop is shown below.

~\\
\begin{lstlisting}[style=defaultNiosStyle]
DO_DELAY: 	movia 	r7, 8000000 		# delay counter
SUB_LOOP: 	subi 	r7, r7, 1
			bne 	r7, zero, SUB_LOOP
\end{lstlisting}

~\\
To avoid ``missing'' any button presses while the processor is executing the delay loop, you
should use the {\it Edgecapture} register in the {\it KEY} port, shown in Figure~\ref{fig:KEY}.
When a pushbutton is pressed, the corresponding bit in the {\it Edgecapture} register is
set to 1, and it remains set until reset to 0 by writing into the register.


Perform the following:

\begin{enumerate}
\item Create a new folder to hold your Monitor Program project for this part. Create a
file called {\it part2.s} and type your assembly language code into this file.

\item
Make a new Monitor Program project in the folder where you stored the {\it part2.s}
file. Select Nios~II as the target processor architecture and use the appropriate pre-built 
computer system for your DE-series board.

\item
Compile, download, and test your program. 
\end{enumerate}

\section*{Part III}
\addcontentsline{toc}{3}{Part III}
In Part II you used a delay loop to cause the Nios~II processor to wait for approximately 0.25 
seconds. The processor loaded a large value into a register before the loop, and then 
decremented that value until it reached 0.  In this part you are to modify your code so that a
hardware timer is used to measure an exact delay of 0.25 seconds. You should use polled I/O to
cause the Nios~II processor to wait for the timer.

~\\
The pre-built computer systems include an {\it Interval Timer} implemented in the FPGA that can be used by
the Nios~II processor. This timer can be loaded with a preset value, and then counts down to 
zero using the 100-MHz clock signal provided as the system clock in the DE-series computer systems (or 50-Mhz for the DE2-115 Computer System). The programming interface 
for the timer includes six 16-bit registers, as illustrated in Figure~\ref{fig:timer}.

\begin{figure}[H]
	\begin{center}
	\includegraphics[scale=1]{figures/figuretimer.pdf}
	\end{center}
	\caption{The Interval Timer registers.}
\label{fig:timer}
\end{figure}

The {\it TO} bit in the {\it Status} register provides a timeout signal which is set to 1 
by the timer when it has reached a count value of zero.  
You should poll this bit in your program to cause the Nios~II processor 
to wait for the timer.  The {\it TO} bit can be reset by writing a 0 into it.  

~\\
The {\it CONT} bit affects the continuous operation of the timer.  When the timer reaches
a count value of zero it automatically reloads the specified starting count value. If 
{\it CONT} is set to 1, then the timer will continue counting down automatically.
But if {\it CONT} $=0$, then the timer will stop after it has reached a count value of 0. 
The ({\it START}/{\it STOP}) bits can be used to commence/suspend the operation of the 
timer by writing a 1 into the respective bit.

~\\
The two 16-bit registers for the Counter start value allow the period of the timer to be changed.
The default setting provided gives a timer period of 125 msec. To achieve this period, the starting value of the count is
100 MHz $\times$ 125 msec $=12.5\times10^6$.

~\\
Make a new folder to hold your Monitor Program project for this part. Create a
file called {\it part3.s} and type your assembly language code into this file.
Make a new Monitor Program project for this part of the exercise, and then compile, download, 
and test your program. 

\section*{Part IV}
\addcontentsline{toc}{4}{Part IV}
In this part you are to write an assembly language program that implements a real-time clock. 
Display the time on the seven-segment displays {\it HEX}$3-0$ in the format 
\red{SS}:\red{DD}, where \red{\it SS} are seconds and \red{\it DD} are hundredths of a second.
Measure time intervals of 0.01 seconds in your program by using polled I/O with the
Interval Timer.  You should be able to stop/run the clock by pressing any pushbutton {\it KEY}.
When the clock reaches \red{59}:\red{99}, it should wrap around to \red{00}:\red{00}.

~\\
Make a new folder to hold your Monitor Program project for this part. Create a
file called {\it part4.s} and type your code into this file.  Make a new Monitor Program 
project for this part of the exercise, and then compile, download, and test your program. 


\input{\CommonDocsPath/copyright.tex}
\end{document}
