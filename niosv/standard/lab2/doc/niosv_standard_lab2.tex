\documentclass[epsfig,10pt,fullpage]{article}

\newcommand{\LabNum}{2}
\newcommand{\CommonDocsPath}{../../../../common/docs}
\input{\CommonDocsPath/preamble.tex}

\begin{document}

\centerline{\huge Computer Organization}
~\\
\centerline{\huge Laboratory Exercise \LabNum}
~\\
\centerline{\large Using Logic Instructions with the Nios\textsuperscript{\textregistered} V Processor}
~\\

Logic instructions are needed in many embedded applications.  Logic instructions are useful 
for manipulation of bit strings and for dealing with data at the bit level where only a 
few bits may be of special interest. They are essential in dealing with input/output tasks.
In this exercise we will consider some typical uses. We will assume that you are using the 
Nios\textsuperscript{\textregistered}~V processor in the DE1-SoC Computer (this
laboratory exercise can be done using any DE-series board that supports Nios~V).

~\\
As described for Laboratory Exercise 1, you may want to develop and debug your Nios~V
programs by using the {\it CPUlator} simulation tool. If you have access to a physical DE1-SoC
board, then you may also want to implement your Nios~V programs on the DE1-SoC hardware by 
using the {\it Monitor Program} application.

\section*{Part I}
\addcontentsline{toc}{1}{Part I}
In this part you will implement a Nios~V assembly-language program that counts the longest 
string of 1's in a word of data. For example, if the word of data is {\sf 0x103fe00f}, then 
the required result is 9.

~\\
Perform the following:

\begin{enumerate}
\item Create a new folder to store your solution for this part of the exercise. Create a
file called {\it part1.s}, and store the assembly language code shown in Figure~\ref{fig:code}
in this file (the {\it part1.s} file is provided as part of the {\it design files} for this 
lab exercise). This code uses an algorithm involving shift and 
AND operations to find the required result---make sure that you understand how this works.

\begin{figure}[H]
\begin{center}
\lstinputlisting[style=defaultNiosVStyle]{../design_files/part1.s}
\end{center}
\caption{Assembly-language program that counts consecutive ones.}
\label{fig:code}
\end{figure}

\newpage
\item
If you are using the {\it Monitor Program}, then make a new project in the folder where you 
stored the {\it part1.s} file. Use the DE1-SoC Computer for your project. If you are using 
{\it CPUlator}, then no project is used---you would just load the file {\it part1.s} into 
the {\it CPUlator} as we discussed in Laboratory Exercise 1.
\item
Compile and load the program. Fix any errors that you encounter (if you mistyped some of
the code). Once the program is loaded into memory in your DE1-SoC board (or the {\it simulated}
memory in {\it CPUlator}), single step through the code to observe the program's operation.
\end{enumerate}

\section*{Part II}
\addcontentsline{toc}{2}{Part II}
Perform the following:
\begin{enumerate}
\item Make a new folder and make a copy of the file {\it part1.s} in that new folder. Give
the new file a name such as {\it part2.s}.
\item In the new file {\it part2.s}, take the code which calculates the 
number of consecutive 1's and make it into a subroutine called {\it ones}. The subroutine 
should use register {\it a0} to receive the input data, and the subroutine should also 
return its result to the main program in register {\it a0}.
\item Add more words in memory starting from the label test\_num. You can add as many
words as you like, but include at least 10 words. To terminate the list include the word 0
at the end---check for this 0 entry in your main program to determine when all of the
items in the list have been processed.
\item In your main program, call the newly-created subroutine in a loop for every word of 
data that you placed in memory. Keep track of the longest string of 1's in any of the words, 
and have this result in register {\it s0} when your program completes execution. 
An outline of the required code is shown in Figure~\ref{fig:code2}.
\item Make sure to use breakpoints or single-stepping in the {\it CPUlator} and/or 
{\it Monitor Program} to observe what happens each time the {\it ones} subroutine is called.
\end{enumerate}

\section*{Part III}
\addcontentsline{toc}{3}{Part III}
For this part you are to repeat the task from Part II, except that you should find the 
longest string of 0's in the data words, instead of the longest string of 1's. Your code
should be mostly the same---just replace the {\it ones} subroutine from Part II with a 
new subroutine called {\it zeros} that finds the longest string of successive 0 bits in a
word.

\section*{Part IV}
\addcontentsline{toc}{3}{Part III}
For this part you are to repeat the task from Part III, except that you should find the 
longest sequence of alternating 0's and 1's in the list of data words. For example, if a
word contains the bits {\sf 101101010001}, then these bits have six alternating 0's and 1's.
Your code should be mostly the same---just replace the {\it zeros} subroutine from Part III 
with a new subroutine called {\it alt} that finds the longest string of alternating 0's
and 1's in a word.

\begin{figure}[H]
\begin{center}
\lstinputlisting[style=defaultNiosVStyle]{../design_files/part2.s}
\end{center}
\caption{Using a subroutine to process a list of words.}
\label{fig:code2}
\end{figure}

\input{\CommonDocsPath/copyright.tex}
\end{document}
